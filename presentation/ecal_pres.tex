\documentclass[9pt]{beamer}

\input{StyleFile.tex}

\usepackage{wrapfig}

\usepackage{tikz}
\usepackage[main=russian,english]{babel}
\usepackage[utf8]{inputenc}
\usepackage[T1,T2A]{fontenc}

\usepackage{xcolor, color, soul}
\sethlcolor{red}

\usepackage{tabularx}
\usepackage{geometry}
\usepackage{wrapfig}

\usepackage{subfig}

\usepackage{movie15}

\usepackage{subfig}
\usepackage{tabularx}
\usepackage{booktabs}
\usepackage{pifont}

\newcommand{\cmark}{\ding{51}}
\newcommand{\xmark}{\ding{55}}
\usepackage{xcolor}

\usepackage[backend=biber,style=numeric,urldate=long,maxbibnames=99,sorting=none]{biblatex}
\addbibresource{refs.bib}

\title[\textbf{}]{Оптимизация размера сенсоров для физики элементарных частиц} % The short title appears at the bottom of every slide, the full title is only on the title page

\author[\textbf{Зиманов Алихан}]
{Зиманов Алихан} % Your name

\institute[\textbf{ВШЭ}] % Your institution as it will appear on the bottom of every slide, may be shorthand to save space
{Исследовательский проект\\
ВКР % Your institution for the title page
}

\jointwork{Научный руководитель Болдырев А.С.}
\conference{Москва, 2024}

%===== Uncomment the following if you wish to use references
%\usepackage[backend=bibtex,citestyle=authoryear-icomp,natbib,maxcitenames=1]{biblatex}
%\addbibresource{bibfile.bib}

% use this so appendices' page numbers do not count
\usepackage{appendixnumberbeamer}

\begin{document}

% Title page, navigation surpressed, no page number
{
\beamertemplatenavigationsymbolsempty
\begin{frame}[plain]
\titlepage
\end{frame}
}

% TOC, navigation surpressed, no page number
{
\beamertemplatenavigationsymbolsempty
\defbeamertemplate*{headline}{miniframes theme no subsection no content}
{ \begin{beamercolorbox}{section in head/foot}
    \vskip\headheight
  \end{beamercolorbox}}
\begin{frame}{Содержание} 
\tableofcontents 
\end{frame} 
}

\addtocounter{framenumber}{-2}

\section{Введение}

\begin{frame}{Постановка задачи}
    \begin{figure}
        \centering
        \includegraphics[width=0.7\textwidth]{images/high_energy_physics.jpg}
    \end{figure}
\end{frame}

\begin{frame}{Актуальность работы}
    \begin{figure}
        \centering
        \subfloat{\includegraphics[width=0.45\textwidth]{../report/graphics/ecal_big.jpg}}\qquad
        \subfloat{\includegraphics[width=0.45\textwidth]{../report/graphics/shower.png}}
        \caption{Электромагнитный калориметр PADME и симуляция электромагнитного ливня с помощью GEANT4.}
    \end{figure}
\end{frame}

\section{Обзор литературы}

\begin{frame}{Смежные работы}
    \begin{block}{Применение глубинного обучения в физике элементарных частиц}
        \begin{itemize}
            \item Идентификация частиц и реконструкция энергии, моделирование калориметрических ливней~\cite{Belayneh_2020}.
            \item Разделение остаточной энергии заряженных и нейтральных частиц, а также сравнение с PFlow алгоритмами~\cite{Di_Bello_2021}.
            \item Реконструкция энергий фотонов с помощью графовых нейронных сетей~\cite{Wemmer_2023}.
        \end{itemize}
    \end{block}
\end{frame}

\begin{frame}{Используемые модели}
    \begin{figure}
        \setcounter{subfigure}{0}
        \centering
        \subfloat[ResNet]{\includegraphics[width=0.45\textwidth]{images/resnet_architecture.png}}\hskip4pt
        \subfloat[Vision Transformer]{\includegraphics[width=0.45\textwidth]{images/vit_architecture.png}}
    \end{figure}

    \begin{block}{}
        ResNet и Vision Transformer являются самыми показательными моделями из области компьютерного зрения.
    \end{block}
\end{frame}

\section{Методология}

\begin{frame}{Данные}
    \begin{block}{}
        Данные сгенерированы с помощью GEANT4. Каждый элемент состоит из:
        \begin{itemize}
            \item Матрица неотрицательных чисел (размерность матрицы  может быть от $10 \times 10$ до $40 \times 40$)
            \item Исходная энергия фотона (от $1$ до $100$ ГэВ)
            \item Положение входной точки фотона (в центральной ячейке калориметра)
        \end{itemize}
    \end{block}

    \begin{figure}
        \centering
        \subfloat{\includegraphics[width=0.38\textwidth]{../report/graphics/data_10x10.png}}\hskip4pt
        \subfloat{\includegraphics[width=0.38\textwidth]{../report/graphics/data_10x10_zoomed.png}}
    \end{figure}
\end{frame}


\begin{frame}{Аугментации}
    \begin{block}{Применяемые аугментации}
        \begin{itemize}
            \item Случайное горизонтальное или вертикальное отражение
            \item Случайный поворот на угол вида $0^{\circ}$, $90^{\circ}$, $180^{\circ}$ и $270^{\circ}$.
        \end{itemize}
    \end{block}

    \begin{figure}
        \setcounter{subfigure}{0}
        \centering
        \subfloat[Оригинал]{\includegraphics[width=0.32\textwidth]{../report/graphics/data_15x15.png}}\hskip2pt
        \subfloat[Отражение]{\scalebox{-1}[1]{\includegraphics[width=0.32\textwidth]{../report/graphics/data_15x15.png}}}\hskip2pt
        \subfloat[Поворот]{\rotatebox[origin=c]{-90}{\includegraphics[width=0.32\textwidth]{../report/graphics/data_15x15.png}}}
    \end{figure}
\end{frame}

\begin{frame}{Модели}
    \begin{block}{Использованные модели}
        \begin{itemize}
            \item Аналитическая модель (AnaModel)
            \item Линейная регрессия (LinReg)
            \item ResNet18
            \item Сверточные сети (CNN)
            \item Vision Transformer (ViT)
        \end{itemize}
    \end{block}

    \begin{block}{}
        Модели обучались одновременно на две задачи: реконструкция энергии и восстановление позиции фотона.
        \begin{figure}
            \centering
            \includegraphics[width=0.35\textwidth]{images/multitask.png}
        \end{figure}
    \end{block}
\end{frame}

\begin{frame}{Метрики}
    \begin{block}{}
        $\{(X_i, E_i, P_i), (\widehat{E}_i, \widehat{P}_i) \}_{i = 1}^{n}$ --- выборка данных, где \begin{itemize}
            \item $X_i$ --- считанные калориметром значения ($\mathbb{R}_{+}^{D \times D}$)
            \item $E_i$ --- исходная энергия фотона ($\mathbb{R}_{+}$)
            \item $P_i = (P_i^x, P_i^y)$ --- позиция входа фотона ($\mathbb{R}^2$)
            \item $\widehat{E}_i$ --- предсказанная энергия ($\mathbb{R}_{+}$)
            \item $\widehat{P}_i = (\widehat{P}_i^x, \widehat{P}_i^y)$ --- предсказанная позиция ($\mathbb{R}^2$)
        \end{itemize}
    \end{block}

    \begin{block}{}
        \begin{align*}
            &\mathcal{L}_{\mathsf{eng}} = \textsf{RMSE/E}(\widehat{E}, E) = \sqrt{\frac{1}{n} \sum_{i = 1}^{n} \left( \frac{\widehat{E}_i - E_i}{E_i} \right)^2} \\
            &\mathcal{L}_{\mathsf{pos}} = \textsf{RMSE}(\widehat{P}, P) = \sqrt{\frac{1}{2 n} \sum_{i = 1}^{n} \left( (\widehat{P}_i^x - P_i^x)^2 + (\widehat{P}_i^y - P_i^y)^2 \right)} \\
            &\mathcal{L}_{\mathsf{total}} = \alpha \cdot \mathcal{L}_{\mathsf{eng}} + (1 - \alpha) \cdot \mathcal{L}_{\mathsf{pos}} , \; \alpha \in [0, 1]
        \end{align*}
    \end{block}
\end{frame}

\section{Результаты}

\begin{frame}{Сравние моделей}
    \begin{figure}
        \centering
        \includegraphics[width=1.0\textwidth]{../report/graphics/models_comp.png}
    \end{figure}

    \begin{block}{}
        Модель ViT показывает лучшие и стабильные результаты.
    \end{block}
\end{frame}

\begin{frame}{Сравние функций потерь}
    \begin{figure}
        \centering
        \includegraphics[width=0.9\textwidth]{../report/graphics/loss_comp.png}
    \end{figure}

    \begin{block}{}
        Обучение на нормализованную (относительную) ошибку приводит к лучшему качеству.
    \end{block}
\end{frame}

\begin{frame}{Отношение важности задач}
    \begin{figure}
        \centering
        \includegraphics[width=1.0\textwidth]{../report/graphics/exp3_alpha_std.png}
    \end{figure}

    \begin{block}{}
        Влияние гиперпараметра $\alpha$ на качество модели\footnote{$\mathcal{L}_{\mathsf{total}} = \alpha \cdot \mathcal{L}_{\mathsf{eng}} + (1 - \alpha) \cdot \mathcal{L}_{\mathsf{pos}}$.}.
    \end{block}
\end{frame}

\begin{frame}{Размер модели}
    \begin{figure}
        \centering
        \includegraphics[width=0.9\textwidth]{../report/graphics/exp4_model_params_std.png}
    \end{figure}

    \begin{block}{}
        Оптимальные параметры модели ViT это 4 слоя, 2 головы и размерность скрытого слоя 16.
    \end{block}
\end{frame}

\begin{frame}{Эффективность аугментаций}
    \begin{table}
        \footnotesize
        \centering
        \begin{tabular}{cc|cccc}
            \toprule
            {} & {} & \multicolumn{4}{c}{\textsf{Размер матрицы}} \\
            \cmidrule(lr){3-6}
            {} & {} & \multicolumn{4}{c}{$\mathsf{15 \times 15}$} \\
            \midrule
            \textsf{Отражения} & \textsf{Повороты} & $\mathcal{L}_{\mathsf{eng}}^{\mathsf{train}}$ & $\mathcal{L}_{\mathsf{pos}}^{\mathsf{train}}$ & $\mathcal{L}_{\mathsf{eng}}^{\mathsf{val}}$ & $\mathcal{L}_{\mathsf{pos}}^{\mathsf{val}}$ \\
            \midrule
            \xmark & \xmark & $\mathsf{0.0182}$ & $\mathsf{0.1456}$ & $\mathsf{0.0217}$ & $\mathsf{0.1535}$ \\
            \cmark & \xmark & $\mathsf{0.0186}$ & $\mathsf{0.1481}$ & $\mathsf{0.0194}$ & $\mathsf{0.1537}$ \\
            \xmark & \cmark & $\mathsf{0.0185}$ & $\mathsf{0.1476}$ & $\mathsf{0.0193}$ & $\mathsf{0.1528}$ \\
            \cmark & \cmark & $\mathsf{0.0185}$ & $\mathsf{0.1475}$ & $\mathsf{0.0192}$ & $\mathsf{0.1525}$ \\        
            
            \bottomrule
        \end{tabular}
    \end{table}

    \begin{block}{Применение аугментаций}
        \begin{itemize}
            \item Улучшение качества на валидационной выборке
            \item Сокращение разрыва между обучающей и валидационной выборкой
        \end{itemize}
    \end{block}
\end{frame}

\begin{frame}{Итоговое качество}
    \begin{figure}
        \centering
        \includegraphics[width=1.0\textwidth]{../report/graphics/best_vit.png}
    \end{figure}

    \begin{block}{Модель ViT одновременно}
        \begin{itemize}
            \item решает задачу восстановления энергии с относительной ошибкой в $1.3\%$
            \item решает задачу реконструкции позиции с точностью, в $20$ раз меньшую размера ячейки калориметра
        \end{itemize}
    \end{block}
\end{frame}

\section{Заключение}

\begin{frame}{Заключение}
    \begin{block}{Результаты работы}
        \begin{itemize}
            \item Исследование и сравнение моделей глубинного обучения
            \item Оптимизация метрик и функций потерь
            \item Анализ влияния размера модели и аугментации данных
            \item Достижение высокой точности реконструкции
            \item Практическое применение и перспективы
        \end{itemize}
    \end{block}
\end{frame}

\begin{frame}[allowframebreaks]
    \frametitle{Список литературы}
    \printbibliography
\end{frame}

\section{Приложения}

\begin{frame}{Метрики}
    \begin{itemize}
        \item Корень из среднеквадратичной ошибки (\textsf{RMSE})
        \item Средняя абсолютная ошибка (\textsf{MAE})
        \item Корень из среднеквадратичной логарифмической ошибки (\textsf{RMSLE}): \[ \textsf{RMSLE}(a, y) = \sqrt{\frac{1}{n} \sum_{i = 1}^{n} (\log(a_i + 1) - \log(y_i + 1))^2} . \]
        \item Взвешенный корень из среднеквадратичной ошибки (\textsf{RMSE/E}): \[ \textsf{RMSE/E}(a, y) = \sqrt{\frac{1}{n} \sum_{i = 1}^{n} \left( \frac{a_i - y_i}{y_i} \right)^2} . \]
        \item Взвешенная средняя абсолютная ошибка (\textsf{MAE/E}): \[ \textsf{MAE/E}(a, y) = \frac{1}{n} \sum_{i = 1}^{n} \frac{|a_i - y_i|}{y_i} . \]
    \end{itemize}
\end{frame}

\begin{frame}{Полное сравнение моделей}
    \begin{figure}
        \centering
        \includegraphics[width=1.0\textwidth]{../report/graphics/models_comp_extra.png}
    \end{figure}

    \begin{block}{}
        Модель CNN показывает слабые результаты, поэтому не была включена в основные слайды.
    \end{block}
\end{frame}

\begin{frame}{Таблица результатов лучшей модели}
    \begin{table}
        \footnotesize
        \centering
        \begin{tabular}{lrrrrrrr}
            \toprule
            {} & \multicolumn{6}{c}{\textsf{Размер матрицы}} \\
            \cmidrule(lr){2-7}
            \textsf{Метрика} & $\mathsf{10 \times 10}$ &  $\mathsf{15 \times 15}$ &  $\mathsf{20 \times 20}$ &  $\mathsf{25 \times 25}$ &  $\mathsf{30 \times 30}$ &  $\mathsf{40 \times 40}$ \\
            \midrule
            $\mathcal{L}_{\mathsf{total}}$ & $\mathsf{0.1153}$ & $\mathsf{0.0852}$ & $\mathsf{0.0702}$ & $\mathsf{0.0588}$ & $\mathsf{0.0535}$ & $\mathsf{0.0453}$ \\
            $\mathcal{L}_{\mathsf{eng}}$ & $\mathsf{0.0189}$ & $\mathsf{0.0189}$ & $\mathsf{0.0194}$ & $\mathsf{0.0187}$ & $\mathsf{0.0189}$ & $\mathsf{0.0190}$ \\
            $\mathcal{L}_{\mathsf{pos}}$ & $\mathsf{0.2117}$ & $\mathsf{0.1515}$ & $\mathsf{0.1211}$ & $\mathsf{0.0989}$ & $\mathsf{0.0881}$ & $\mathsf{0.0715}$ \\
            \textsf{Размер одной ячейки} & $\mathsf{6.0600}$ & $\mathsf{4.0400}$ & $\mathsf{3.0300}$ & $\mathsf{2.4240}$ & $\mathsf{2.0200}$ & $\mathsf{1.5150}$ \\
            $\textsf{MAE/E}_{\textsf{eng}}$ & $\mathsf{0.0131}$ & $\mathsf{0.0127}$ & $\mathsf{0.0132}$ & $\mathsf{0.0124}$ & $\mathsf{0.0130}$ & $\mathsf{0.0129}$ \\        
            \bottomrule
        \end{tabular}
    \end{table}

    \begin{block}{}
        Модель \textsf{ViT} способна решать задачу реконструкции позиции с точностью, в $20$ раз меньшую чем длина стороны центральной ячейки. Более того, данная модель решает задачу восстановления энергии с относительной ошибкой в $1.3\%$.
    \end{block}
\end{frame}

\end{document}