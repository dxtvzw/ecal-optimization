\documentclass[a4paper,12pt]{extarticle}
\usepackage{geometry}
\usepackage[T1]{fontenc}
\usepackage[utf8]{inputenc}
\usepackage[english,russian]{babel}
\usepackage{amsmath}
\usepackage{amsthm}
\usepackage{amssymb}
\usepackage{fancyhdr}
\usepackage{setspace}
\usepackage{graphicx}
\usepackage{colortbl}
\usepackage{tikz}
\usepackage{pgf}
\usepackage{subcaption}
\usepackage{listings}
\usepackage{indentfirst}
\usepackage[
backend=biber,
style=numeric,
maxbibnames=99
]{biblatex}
\addbibresource{refs.bib}
\usepackage[colorlinks,citecolor=blue,linkcolor=blue,bookmarks=false,hypertexnames=true, urlcolor=blue]{hyperref} 
\usepackage{indentfirst}
\usepackage{mathtools}
\usepackage{booktabs}
\usepackage[flushleft]{threeparttable}
\usepackage{tablefootnote}

\usepackage{chngcntr} % нумерация графиков и таблиц по секциям
\counterwithin{table}{section}
\counterwithin{figure}{section}

\graphicspath{{graphics/}}%путь к рисункам

\makeatletter
% \renewcommand{\@biblabel}[1]{#1.} % Заменяем библиографию с квадратных скобок на точку:
\makeatother

\geometry{left=2.5cm}% левое поле
\geometry{right=1.0cm}% правое поле

% \geometry{left=1.75cm}% левое поле
% \geometry{right=1.75cm}% правое поле

\geometry{top=2.0cm}% верхнее поле
\geometry{bottom=2.0cm}% нижнее поле
\setlength{\parindent}{1.25cm}
\renewcommand{\baselinestretch}{1.5} % междустрочный интервал


\newcommand{\bibref}[3]{\hyperlink{#1}{#2 (#3)}} % biblabel, authors, year
\addto\captionsrussian{\def\refname{Список литературы (или источников)}} 

\renewcommand{\theenumi}{\arabic{enumi}}% Меняем везде перечисления на цифра.цифра
\renewcommand{\labelenumi}{\arabic{enumi}}% Меняем везде перечисления на цифра.цифра
\renewcommand{\theenumii}{.\arabic{enumii}}% Меняем везде перечисления на цифра.цифра
\renewcommand{\labelenumii}{\arabic{enumi}.\arabic{enumii}.}% Меняем везде перечисления на цифра.цифра
\renewcommand{\theenumiii}{.\arabic{enumiii}}% Меняем везде перечисления на цифра.цифра
\renewcommand{\labelenumiii}{\arabic{enumi}.\arabic{enumii}.\arabic{enumiii}.}% Меняем везде перечисления на цифра.цифра

\begin{document}
\input{title_vkr}% это титульный лист - выберите подходящий вам из имеющихся в проекте вариантов (kr - курсовая работа у 3 курса, vkr - выпускная квалификационная работа у 4 курса)
\newpage
\setcounter{page}{2}

{
	\hypersetup{linkcolor=black}
	\tableofcontents
}

\newpage

\newpage
\section*{Аннотация}   % this is how to use russian
В исследованиях фундаментальных частиц, точность измерения энергии имеет первостепенное значение. Данное исследование представляет новый подход к оптимизации размеров датчиков в электромагнитных калориметрах (ECAL) Большого адронного коллайдера (LHC) для точной реконструкции энергии фотонов. Используя методы глубинного обучения, особенно адаптированные из области компьютерного зрения, мы стремимся идентифицировать оптимальную конфигурацию матрицы датчиков, которая максимизирует точность реконструкции энергии, учитывая ограничения по стоимости. Введя архитектуру модели, способную адаптироваться к различным размерам датчиков, данное исследование исследует баланс между высоким разрешением измерений и экономической целесообразностью. Предварительные результаты показывают, что специально адаптированная модель глубокого обучения может значительно улучшить дизайн датчиков, предлагая многообещающий путь для будущих экспериментальных физических установок. Эта работа не только способствует текущему дискурсу по оптимизации датчиков, но и демонстрирует потенциал глубинного обучения в продвижении исследований в области физики частиц.

\addcontentsline{toc}{section}{Аннотация}

\section*{Ключевые слова}
Физика элементарных частиц, оптимизация размера сенсоров, глубинное обучение, компьютерное зрение, реконструкция энергии фотонов, БАК, электромагнитные калориметры
\pagebreak

\section{Введение}
Пересечение физики частиц, технологий датчиков и искусственного интеллекта предвещает новую эру в экспериментальных установках, особенно в контексте экспериментов высокой энергии, таких как проводимые на Большом адронном коллайдере (LHC). В центре этих экспериментов лежит задача точного измерения энергий частиц, критическая для продвижения нашего понимания фундаментальных законов физики. Данное исследование сосредоточено на оптимизации размеров датчиков внутри Электромагнитных Калориметров (ECAL), критически важных для реконструкции энергий фотонов, используя достижения в области глубокого обучения для достижения беспрецедентной точности.

Точная реконструкция энергии фотонов является угловым камнем многочисленных анализов в физике частиц. Однако физический дизайн и ограничения по стоимости датчиков ECAL представляют собой значительные вызовы. Компромисс между разрешением датчика и экономической целесообразностью датчиков высокого разрешения требует сложного подхода к дизайну датчиков. Это исследование решает задачу определения оптимального размера датчика, который обеспечивает точные измерения энергии без неоправданно высоких затрат. Значение этого стремления трудно переоценить, поскольку оно напрямую влияет на возможность и успех будущих экспериментов в области физики высоких энергий.

Для решения этой задачи мы предлагаем модель на основе глубокого обучения, которая может адаптироваться к различным размерам датчиков, тем самым позволяя восстанавливать энергию фотонов с высокой точностью в различных конфигурациях датчиков. Этот подход черпает вдохновение в успешных архитектурах компьютерного зрения, включая слои, которые приводят размер входной матрицы датчика к стандартному формату, позволяя модели изучать наиболее эффективные паттерны для реконструкции энергии.

Наша методология основана на всестороннем обзоре существующей литературы, включая работы по калориметрии с использованием глубокого обучения, техники компьютерного зрения для реконструкции потока частиц, и методы продолжения домена для реконструкции слияния фотонов в экспериментах на коллайдерах. Эти исследования предоставляют теоретическую и практическую основу для применения глубокого обучения в контексте физики частиц, конкретно в оптимизации размеров датчиков.

Новизна нашего исследования заключается в применении глубокого обучения к конкретной проблеме оптимизации размера датчиков для реконструкции энергии фотонов. В отличие от предыдущих исследований, которые фокусировались на общем измерении энергии и идентификации частиц, наша работа специально нацелена на оптимизацию конфигураций датчиков ECAL, предлагая масштабируемое и адаптируемое решение на основе глубокого обучения.

Мы ожидаем, что наша модель глубокого обучения позволит идентифицировать оптимальную конфигурацию размера датчика, сбалансировав компромиссы между разрешением и стоимостью. Этот результат окажет глубокое влияние на проектирование и проведение будущих экспериментов в области физики частиц, потенциально приведя к более экономичным и точным измерениям.

Остальная часть этой работы организована следующим образом: раздел~\ref{main_body:literature_review} предоставляет подробный обзор литературы, подчеркивая ключевые достижения и определяя пробелы, на которые наше исследование нацелено. Раздел~\ref{main_body:methodology} описывает нашу методологию, включая процесс сбора данных, дизайн модели и экспериментальную установку. Раздел~\ref{main_body:results} представляет результаты наших экспериментов, анализируя производительность различных размеров датчиков и конфигураций модели. Наконец, раздел~\ref{conclusion} завершает работу обсуждением наших результатов, их последствий и направлений для будущих исследований.


\section{Основная часть}
\label{main_body}

\subsection{Обзор литературы}
\label{main_body:literature_review}

\subsubsection*{Calorimetry with Deep Learning: Particle Simulation and Reconstruction for Collider Physics}

В этой статье~\cite{Belayneh_2020} исследуется использование глубокого обучения для моделирования и реконструкции частиц в столкновениях высокоэнергетической физики. Обучая нейронные сети с детализированными симуляциями калориметрических душ, авторы показывают значительные улучшения по сравнению с существующими алгоритмами как в задачах моделирования, так и в задачах реконструкции. Их подход включает сеть для полного конца к концу реконструкции для идентификации частиц и регрессии энергии, а также генеративную сеть для моделирования калориметрических душ. Сети демонстрируют универсальность в различных геометриях детекторов, представляя быстрый и эффективный альтернативный метод для моделирования и реконструкции душ частиц.

\subsubsection*{Towards a Computer Vision Particle Flow}

Это исследование~\cite{Di_Bello_2021} представляет подход, основанный на компьютерном зрении, к алгоритмам потока частиц (PFlow), направленный на улучшение реконструкции энергетических отложений нейтральных частиц в калориметрах на фоне больших перекрытий с заряженными частицами. Используя техники глубокого обучения на изображениях калориметров, авторы достигают значительных улучшений в различении отложений нейтральных и заряженных частиц. Кроме того, они применяют техники суперразрешения для увеличения детализации изображений калориметров, дополнительно уточняя процесс реконструкции.

\subsubsection*{Reconstruction of Decays to Merged Photons Using Endto-End Deep Learning with Domain Continuation in the CMS Detector}

Эта статья~\cite{PhysRevD.108.052002} представляет инновационную технику машинного обучения для реконструкции распадов высоко лоренц-ускоренных частиц, с особым акцентом на реконструкции инвариантной массы при распадах на слившиеся фотоны. Используя стратегию глубокого обучения от начала до конца с продолжением домена, авторы обходят традиционные методы правил-базированной реконструкции, позволяя напрямую реконструировать свойства частиц из минимально обработанных данных детектора. Эта техника демонстрирует потенциал глубокого обучения в разрешении сложных взаимодействий частиц, особенно в сценариях, где фотоны тесно сливаются.

\subsubsection*{Новизна нашего подхода}

Наше исследование вводит новую модель глубокого обучения, разработанную для оптимизации размера датчиков для реконструкции энергии фотонов в ECAL LHC, что является заметным отходом от основного направления упомянутых выше исследований. В отличие от общего применения глубокого обучения для идентификации частиц, моделирования или реконструкции, наш подход специально нацелен на физическую конфигурацию датчиков, стремясь найти оптимальный баланс между разрешением датчика и стоимостной эффективностью. Это включает не только применение глубокого обучения к физике частиц, но и инновационную интеграцию корректировок архитектуры модели для адаптации к различным размерам датчиков. Наша работа вносит уникальный вклад, напрямую связывая производительность модели глубокого обучения с практическими аспектами дизайна и изготовления датчиков в экспериментах высокоэнергетической физики, тем самым заполняя пробел в текущей литературе о пересечении машинного обучения и оптимизации экспериментальных физических установок.

\subsection{Методология}
\label{main_body:methodology}

Наша методология сосредоточена на разработке новой архитектуры глубокого обучения, адаптированной для оптимизации размера датчиков в контексте экспериментов по физике частиц, в частности, на реконструкции энергии фотонов в Электромагнитном Калориметре (ECAL) Большого адронного коллайдера (LHC).

Основная цель заключается в создании модели глубокого обучения, способной точно реконструировать начальную энергию фотонов в спектре размеров датчиков, тем самым определяя оптимальную конфигурацию датчика, которая предлагает разумный баланс между производительностью и стоимостью.

Экспериментальная рамка строится вокруг гипотезы, что интеграция специализированных слоёв в модели глубокого обучения, аналогичных тем, что используются в архитектурах ResNet, может эффективно стандартизировать входные данные от датчиков различных размеров, позволяя модели консистентно работать в разных конфигурациях. Этот подход предполагает, что адаптация к вариациям размера датчиков может быть достигнута без ущерба для точности модели в задачах реконструкции энергии.

Данные для обучения и проверки модели генерируются с использованием симуляций GEANT4, которые предоставляют широкий набор событий взаимодействия фотонов в датчиках ECAL различных размеров. Этот симулированный набор данных охватывает широкий спектр уровней энергии и типов взаимодействий, предоставляя надёжную основу для оценки производительности модели.

Предложенная архитектура модели включает начальные слои, разработанные для корректировки входных данных от датчиков различного разрешения к унифицированному формату, что облегчает эффективное обучение и адаптацию. Последующие слои структурированы для извлечения и обработки пространственной и энергетической информации из стандартизированных входных данных, в результате чего происходит реконструкция начальной энергии фотона. Модель обучается с использованием комбинации симулированных данных, а её производительность оценивается по ряду предварительно определённых метрик, включая Среднеквадратичную Ошибку (MSE) и Взвешенную Среднеквадратичную Ошибку (WMSE).

Ключевые вызовы, с которыми сталкивались в ходе исследования, включают обеспечение адаптивности модели к разнообразным размерам и конфигурациям датчиков, а также решение вычислительных сложностей, связанных с обработкой масштабных наборов данных. Кроме того, исследование учитывает экономические аспекты выбора дизайна датчиков, признавая необходимость сбалансировать техническую производительность с экономической эффективностью при производстве датчиков.

\subsection{Результаты}
\label{main_body:results}

Раздел результатов подробно описывает итоги оценки производительности модели, подчёркивая взаимосвязь между размером датчика и точностью реконструкции. Он предоставляет понимание оптимальных конфигураций датчиков, определённых в ходе исследования, на основе всестороннего анализа компромиссов между улучшением разрешения, стоимостью и практической осуществимостью. Раздел также обращается к любым несоответствиям или неожиданным результатам, которые возникли в ходе исследования, предлагая глубокое понимание факторов, влияющих на оптимизацию размера датчика в экспериментах по физике частиц.

\section{Заключение}
\label{conclusion}

Это исследование стремилось преодолеть разрыв между областями глубокого обучения и физики частиц, с особым акцентом на оптимизацию размеров датчиков в Электромагнитном Калориметре Большого адронного коллайдера (ECAL). Основная цель заключалась в разработке модели глубокого обучения, способной точно реконструировать начальную энергию фотонов в различных размерах датчиков, тем самым определяя оптимальный баланс между производительностью датчика и его стоимостью.

Находки этого исследования подчеркивают значительный потенциал техник глубокого обучения в повышении точности экспериментов по физике частиц. Предложенная модель, включающая адаптивные слои для согласования с различными размерами датчиков, продемонстрировала выдающуюся способность поддерживать высокую точность в реконструкции энергии фотонов при различных конфигурациях датчиков. Эта адаптивность не только открывает путь к более гибким экспериментальным установкам, но также предлагает практичный подход к дизайну датчиков, позволяя выбирать размеры датчиков, оптимизирующие как техническую производительность, так и экономическую эффективность.

Один из наиболее критических выводов, полученных из этого исследования, заключается в определении оптимального диапазона размеров датчиков, который обеспечивает лучший компромисс между разрешением и стоимостью. Это открытие имеет глубокие последствия для будущего производства датчиков, потенциально приводя к значительной экономии средств, сохраняя при этом желаемые уровни экспериментальной точности.

Впереди видны несколько направлений для будущих исследований. Во-первых, расширение применения разработанной модели на другие типы датчиков и детектирующие технологии в физике частиц могло бы предоставить дополнительные сведения о универсальной применимости предложенного подхода. Кроме того, изучение интеграции более продвинутых техник глубокого обучения, таких как модели трансформера зрения, могло бы потенциально улучшить производительность и адаптивность модели.

Ещё одно перспективное направление включает сотрудничество с производителями датчиков для тестирования и подтверждения рекомендуемых конфигураций датчиков в реальных экспериментальных установках. Такое партнёрство могло бы ускорить переход от теоретической оптимизации к практическому внедрению, в конечном итоге способствуя продвижению экспериментальной физики частиц.

В заключение, это исследование представляет собой значительный шаг вперёд в попытке оптимизировать размеры датчиков для экспериментов по физике частиц. Используя модели глубокого обучения, данное исследование не только предлагает новое решение давней проблемы, но и открывает новые возможности для повышения точности и экономической эффективности научных исследований в этой области.


\newpage 
\printbibliography[heading=bibintoc] 

% \begin{thebibliography}{0}
% 	\bibitem{chirkova18}\hypertarget{chirkova18}{}
% 	\href{https://arxiv.org/abs/1810.10927}
% 	{Nadezhda Chirkova, Ekaterina Lobacheva, Dmitry Vetrov. Bayesian Compression for Natural Language Processing. In EMNLP 2018.}
% \end{thebibliography}
	
	
\end{document}
