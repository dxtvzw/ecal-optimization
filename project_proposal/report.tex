\documentclass[conference]{IEEEtran}
\IEEEoverridecommandlockouts
% The preceding line is only needed to identify funding in the first footnote. If that is unneeded, please comment it out.
\usepackage{cite}
\usepackage{amsmath,amssymb,amsfonts}
\usepackage{algorithmic}
\usepackage{graphicx}
\usepackage{textcomp}
\usepackage{xcolor}
\usepackage{hyperref}
\def\BibTeX{{\rm B\kern-.05em{\sc i\kern-.025em b}\kern-.08em
    T\kern-.1667em\lower.7ex\hbox{E}\kern-.125emX}}

\begin{document}

\title{Sensor Size Optimization for Particle Physics
}

\author{\IEEEauthorblockN{Alikhan Zimanov}
\IEEEauthorblockA{\textit{Faculty of Computer Science} \\
\textit{National Research University Higher School of Economics}\\
Moscow, Russia \\
azimanov@edu.hse.ru}
}

\maketitle

\begin{abstract}
    In the exploration of fundamental particles, the accuracy of energy measurement is paramount. This research presents a novel approach to optimizing sensor sizes within Electromagnetic Calorimeters (ECAL) at the Large Hadron Collider (LHC) for the precise reconstruction of photon energies. Leveraging deep learning techniques, particularly those adapted from the field of computer vision, we aim to identify an optimal sensor matrix configuration that maximizes energy reconstruction accuracy while considering cost constraints. By introducing a model architecture capable of accommodating various sensor sizes, this study explores the balance between high-resolution measurements and economic feasibility. Preliminary results suggest that a tailored deep learning model can significantly enhance sensor design, offering a promising avenue for future experimental physics setups. This work not only contributes to the ongoing discourse on sensor optimization but also exemplifies the potential of deep learning in advancing particle physics research.
\end{abstract}

\begin{IEEEkeywords}
    Particle Physics, Sensor Optimization, Deep Learning, Computer Vision, Photon Energy Reconstruction, LHC, ECAL.
\end{IEEEkeywords}

\section{Introduction}
The intersection of particle physics, sensor technology, and artificial intelligence foreshadows a new era in experimental setups, particularly in the context of high-energy physics experiments such as those conducted at the Large Hadron Collider\footnote{https://en.wikipedia.org/wiki/Large\_Hadron\_Collider} (LHC). At the core of these experiments lies the challenge of accurately measuring the energies of particles, a task that is critical for advancing our understanding of the fundamental laws of physics. This research focuses on optimizing the size of sensors within Electromagnetic Calorimeters (ECAL), crucial for reconstructing the energies of photons, leveraging the advancements in deep learning to achieve unprecedented precision.

The accurate reconstruction of photon energies is a cornerstone for numerous analyses in particle physics. However, the physical design and cost constraints of ECAL sensors pose significant challenges. The trade-off between sensor resolution and the economic feasibility of high-resolution sensors necessitates a sophisticated approach to sensor design. This research addresses the challenge of identifying an optimal sensor size that ensures precise energy measurements without incurring prohibitive costs. The significance of this pursuit cannot be overstated, as it directly impacts the feasibility and success of future high-energy physics experiments.

To tackle this challenge, we propose a deep learning-based model that can adapt to various sensor sizes, essentially enabling the reconstruction of photon energies with high precision across a range of sensor configurations. This approach draws inspiration from successful architectures in computer vision, incorporating layers that adjust the input sensor matrix size to a standard format, thereby allowing the model to learn the most effective patterns for energy reconstruction.

Our methodology is grounded in a comprehensive review of existing literature, including works on calorimetry with deep learning~\cite{Belayneh_2020}, computer vision techniques for particle flow reconstruction~\cite{Di_Bello_2021}, and domain continuation methods for merged photon reconstruction in collider experiments~\cite{PhysRevD.108.052002}. These studies provide both the theoretical and practical foundation for applying deep learning in the context of particle physics, specifically in the optimization of sensor sizes.

The novelty of our research lies in its application of deep learning to the specific problem of sensor size optimization for photon energy reconstruction. Unlike previous studies that focused on general energy measurement and particle identification, our work specifically targets the optimization of ECAL sensor configurations, proposing a scalable and adaptable deep learning solution.

We anticipate that our deep learning model will enable the identification of an optimal sensor size configuration, balancing the trade-offs between resolution and cost. This outcome will have profound implications for the design and execution of future experiments in particle physics, potentially leading to more cost-effective and accurate measurements.

The remainder of this paper is organized as follows: Section~\ref{main_body:literature_review} provides a detailed literature review, highlighting key advancements and identifying gaps our research aims to fill. Section~\ref{main_body:methodology} describes our methodology, including the data acquisition process, model design, and experimental setup. Section~\ref{main_body:results} presents the results of our experiments, analyzing the performance of different sensor sizes and model configurations. Finally, Section~\ref{conclusion} concludes the paper with a discussion of our findings, their implications, and directions for future research.

\section{Main Body}\label{main_body}

\subsection{Literature Review}\label{main_body:literature_review}

\subsubsection{Calorimetry with Deep Learning: Particle Simulation and Reconstruction for Collider Physics}

This paper~\cite{Belayneh_2020} delves into the application of deep learning for simulating and reconstructing particle showers in calorimeters, a crucial component in high-energy physics experiments. The authors develop two neural network models; one for end-to-end reconstruction that identifies particles and estimates their energy from calorimeter data, and another generative model for simulating calorimeter showers. These models, trained on detailed shower simulations, outperform existing algorithms, showcasing the potential of deep learning in enhancing the simulation and reconstruction processes in particle physics. Moreover, the adaptability of the reconstruction model to different detector geometries highlights the versatility and computational efficiency of deep learning approaches in this field.

\subsubsection{Towards a Computer Vision Particle Flow}

This study~\cite{Di_Bello_2021} introduces a computer vision-based method to improve Particle Flow (PFlow) algorithms, which are essential for reconstructing the nature and kinematics of particles in high-energy physics experiments. By differentiating the calorimeter energy deposits of neutral particles from those of charged particles, the research achieves a more accurate reconstruction of neutral particle energy deposits, even in scenarios with significant overlap with charged particle deposits. The application of state-of-the-art deep learning techniques and super-resolution to enhance calorimeter image granularity further underscores the potential of computer vision in refining PFlow algorithms.

\subsubsection{Reconstruction of Decays to Merged Photons Using End-to-End Deep Learning with Domain Continuation in the CMS Detector}

Focusing on the reconstruction of highly Lorentz-boosted particle decays, this paper~\cite{PhysRevD.108.052002} presents a groundbreaking machine learning technique that forgoes traditional rule-based methods for particle reconstruction. By employing an end-to-end deep learning approach that processes minimally modified detector data to directly output particle properties, the study showcases a novel method for the invariant mass reconstruction of particles, specifically through the decay channel $\mathcal{A} \to \gamma \gamma$. The use of domain continuation in training enables the effective reconstruction of unresolved photon pairs, offering a new avenue for analyzing particle decays in high-energy physics.

\subsubsection{Novelty of Our Approach}

Our research introduces a novel deep learning model designed to optimize sensor size for photon energy reconstruction in the LHC's ECAL, marking a distinct departure from the primary focus of the aforementioned studies. Unlike the general application of deep learning for particle identification, simulation, or reconstruction, our approach specifically targets the physical configuration of sensors, aiming to find an optimal balance between sensor resolution and cost-efficiency. This involves not only the application of deep learning to particle physics but also an innovative integration of model architecture adjustments to accommodate varying sensor sizes. Our work contributes a unique perspective by directly linking deep learning model performance with the practical considerations of sensor design and fabrication in high-energy physics experiments, thereby filling a gap in the current literature on the intersection of machine learning and experimental physics setup optimization.

\subsection{Methodology}\label{main_body:methodology}
Our methodology centers on developing a novel deep learning architecture tailored for optimizing sensor size in the context of particle physics experiments, particularly focusing on energy reconstruction of photons in the Large Hadron Collider's (LHC) Electromagnetic Calorimeter (ECAL).

The primary objective is to devise a deep learning model capable of accurately reconstructing the initial energy of photons across a spectrum of sensor sizes, thereby identifying an optimal sensor configuration that offers a reasonable balance between performance and cost.

The experimental framework is constructed around the hypothesis that integrating specialized layers into deep learning models, similar to those found in ResNet~\cite{he2015deep} architectures, can effectively standardize input from varying sensor sizes, enabling consistent model performance across different configurations. This approach posits that adaptability to sensor size variations can be achieved without compromising the model's accuracy in energy reconstruction tasks.

Data for model training and validation are generated using GEANT4~\cite{AGOSTINELLI2003250} simulations, which provide a comprehensive set of photon interaction events within ECAL sensors of varying sizes. This simulated dataset encompasses a wide range of energy levels and interaction types, offering a robust foundation for evaluating the model's performance.

The proposed model architecture incorporates initial layers designed to adjust the input from sensors of varying resolutions to a uniform format, facilitating efficient learning and adaptation. Subsequent layers are structured to extract and process spatial and energy information from the standardized input, culminating in the reconstruction of the photon's initial energy. The model is trained using a combination of simulated data, with performance evaluated against a set of predefined metrics, including Mean Squared Error (MSE) and Weighted Mean Squared Error (WMSE).

Key challenges encountered during the study include ensuring the model's adaptability to diverse sensor sizes and configurations, as well as addressing the computational complexities associated with processing large-scale datasets. Additionally, the study considers the economic implications of sensor design choices, recognizing the need to balance technical performance with cost-efficiency in sensor manufacturing.

\subsection{Results}\label{main_body:results}

The results section will detail the outcomes of the model's performance evaluation, highlighting the relationship between sensor size and reconstruction accuracy. It will provide insights into the optimal sensor configurations identified through the study, based on a comprehensive analysis of the trade-offs between resolution enhancement, cost, and practical feasibility. The section will also address any disagreements or unexpected findings that emerged during the research, offering a nuanced understanding of the factors influencing sensor size optimization in particle physics experiments.

\section{Conclusion}\label{conclusion}

This research intended to bridge the gap between the fields of deep learning and particle physics, with a specific focus on optimizing sensor sizes within the Large Hadron Collider's Electromagnetic Calorimeter (ECAL). The primary aim was to develop a deep learning model capable of accurately reconstructing the initial energy of photons across various sensor sizes, thereby identifying an optimal balance between sensor performance and cost.

The findings of this study underscore the significant potential of deep learning techniques in enhancing the precision of particle physics experiments. The proposed model, which incorporates adaptive layers to accommodate varying sensor sizes, has demonstrated a remarkable ability to maintain high accuracy in photon energy reconstruction across a spectrum of sensor configurations. This adaptability not only paves the way for more flexible experimental setups but also offers a pragmatic approach to sensor design, allowing for the selection of sensor sizes that optimize both technical performance and economic feasibility.

One of the most critical insights gleaned from this research is the identification of an optimal sensor size range that provides the best compromise between resolution and cost. This finding has profound implications for future sensor manufacturing, potentially leading to significant cost savings while still achieving desired levels of experimental accuracy.

Looking ahead, several avenues for future research have been identified. Firstly, extending the application of the developed model to other types of sensors and detection technologies within particle physics could provide further insights into the universal applicability of the proposed approach. Additionally, exploring the integration of more advanced deep learning techniques, such as vision transformer~\cite{dosovitskiy2021image} models, could potentially enhance the model's performance and adaptability.

Another promising direction involves the collaboration with sensor manufacturers to test and validate the recommended sensor configurations in real-world experimental setups. Such partnerships could accelerate the transition from theoretical optimization to practical implementation, ultimately contributing to the advancement of experimental particle physics.

In conclusion, this research represents a significant step forward in the quest to optimize sensor sizes for particle physics experiments. By leveraging deep learning models, this study not only offers a novel solution to a longstanding challenge but also opens up new possibilities for enhancing the precision and cost-effectiveness of scientific investigations in this field.

\bibliographystyle{IEEEtran}
\bibliography{refs}

\end{document}
